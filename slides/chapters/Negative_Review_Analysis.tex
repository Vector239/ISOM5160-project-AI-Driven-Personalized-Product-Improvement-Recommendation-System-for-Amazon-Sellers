\section{Negative Review Analysis}

\begin{frame}{Negative Review Analysis}
    
	\begin{block}{Why need negative review analysis?}
		\begin{itemize}
			\item Assist in finding improvement directions(By knowing the distribution of negative reviews, decisions can be made to improve like supply chain, operations, preservation, transportation, and other aspects)
			\item Analyze the problems with the product(By knowing the distribution of negative reviews, decisions can be made to improve supply chain, operations, preservation, transportation, and other aspects)
		\end{itemize}
	\end{block}
    
	\begin{block}{Steps}
		\begin{itemize}
			\item Read the comment data after data cleaning
			\item Filter travel reviews through logical judgment
			\item Classify and analyze negative reviews based on keywords
		\end{itemize}
	\end{block}

\end{frame}

\begin{frame}{Negative Review Analysis}
    
	\begin{block}{Negative evaluation criteria}
		\scriptsize
		\begin{itemize}
			\item Accurately locate real negative reviews through dual filtering criteria
			\item Objective rating: Score <= 2 (considered low on a 5-point scale)
			\item Subjective sentiment: Sentimentscore < 0.4 (sentiment tendency score generated by data\_cleaning.py, below 0.4 is considered negative)
		\end{itemize}
	\end{block}
    
	\begin{block}{Keyword filtering}
		\scriptsize
		\begin{itemize}
			\item Map high-frequency keywords to 8 common negative cause categories in the food industry, and define the classification rules through the category\_mapping dictionary:
			\item Taste: such as taste, bitter, bland, etc
			\item Quality: such as poor, cheap, terrible, etc
			\item Expired (expired/spoiled): such as expired, rotten, mold, etc
			\item Other categories: packaging, price, delivery, quantity, smell
			\item Keywords that do not match the preset category are classified as' other ', ensuring that all keywords are categorized accordingly
		\end{itemize}
	\end{block}

\end{frame}


















